% MDMPROP.TEX -- MDM electronic proposal form.
% Revision: October 7, 2016 --- valid for the 2016A observing semester.
% DEADLINE: 5:00 PM Monday November 7, 2016
%
% General Instructions:
%
%   I. Where/how/when to submit this form:
%
%      Give a paper copy to Jules or his mailbox on the 10th floor
%      of Pupin by the above deadline, or email the postscript or
%      pdf file to jules@astro.columbia.edu.
%
% THE FORM STARTS HERE

% please don't modify or delete this line.
\documentstyle[mdmprop,11pt]{article}

\newcommand{\Kepler}{{\it Kepler}}
\newcommand{\kepler}{{\it Kepler}}
\newcommand{\gaia}{{\it Gaia}}
\newcommand{\Gaia}{{\it Gaia}}
\newcommand{\TGAS}{{\it TGAS}}
\newcommand{\tgas}{{\it TGAS}}

\begin{document}

% Give a descriptive title for the proposal in the \title command.
%
%    \title{TEXT}
%
% Note that a title can be quite long; LaTeX will break the title into
% separate lines automatically.  If you wish to indicate line breaks
% yourself, do so with a `\\' command at the appropriate point in
% the title text.

\title{Radial velocity follow-up of Gaia wide-binary candidates in the
\Kepler\ field}

% Investigator's (PI and CoI) information blocks.
%
% The PI must be affiliated with Columbia University.
% Please give each investigator's name, email address.
% At least one investigator must be listed as definitely
% being present at the telescope.
%
%    \name{OBSERVER NAME}
%    \emailaddress{EMAIL ADDRESS}
%    \atthetelescope{Y/N}
%
% DO NOT remove the \begin{PI} and \end{PI} or the \begin{CoI} and
% \end{CoI} lines.

\begin{PI}
\name{Ruth Angus}			% REQUIRED
\emailaddress{ruthangus@gmail.com}		% REQUIRED
\atthetelescope{Y}       % REQUIRED
\end{PI}

\begin{CoI}
\name{Semyeong Oh}			% REQUIRED
\emailaddress{semyeong@astro.princeton.edu}		% REQUIRED
\atthetelescope{N}       % REQUIRED
\end{CoI}

\begin{CoI}
\name{Adrian Price-Whelan}			% REQUIRED
\emailaddress{adrianmpw@gmail.com}
\atthetelescope{N}       % REQUIRED
\end{CoI}

\begin{CoI}
\name{Marcel Agueros}			% REQUIRED
\emailaddress{marcel@astro.columbia.edu}
\atthetelescope{N}       % REQUIRED
\end{CoI}

% \begin{CoI}
% \name{}			% REQUIRED
% \emailaddress{}		% REQUIRED
% \atthetelescope{}       % REQUIRED
% \end{CoI}

% You can supply more CoI blocks, but only the first three will be
% printed on the form.

% Give a general abstract of the scientific justification appropriate for
% a non-specialist.  Write the abstract between the \begin{abstract} and
% \end{abstract} lines.  Limit yourself to approximately 175 words.

% DO NOT remove the \begin{abstract} and \end{abstract} lines.

\begin{abstract}
\end{abstract}

% Indicate whether you request classical scheduling or proxy queue
% observations by putting a C or Q inside the \queue{} curly braces.
% Queue is only on the 2.4m, and only using OSMOS.

\queue{Q}

% Indicate whether this proposal is part of your PhD thesis work by
% putting a Y or an N inside the \thesis{} curly braces.
% (This information is no longer requested).

%\thesis{}

% Indicate whether you are requesting long-term status by putting
% a Y or an N inside the \longterm{} curly braces.  If you do want
% long-term status (a "Y" answer), please tell us the number of nights,
% number of semesters, and what telescopes by filling in \longtermdetails{}.
% Please be brief; \longtermdetails is limited to one line.

\longterm{N}
% \longtermdetails{}

% List the details of the observing runs being requested, for
% UP TO THREE runs.  The parameters for each run are segregated
% between \begin{obsrun} and \end{obsrun} lines.  Please be sure
% that the information is isolated properly for each run.
%
%   \telescope{}	For example, \telescope{1.3m}
%   \instrument{}	For example, \instrument{Mk III + Charlotte}
%   \lunardays{}	For example, \lunardays{14}
%   \optimaldates{}	For example, \optimaldates{Sep 1 - Nov 30}
%   \acceptabledates{}	For example, \acceptabledates{Sep 1 - Dec 15}
%
%
% Instrument combinations may be specified with "+".
%
% \numnights should give the number of nights of the run
% \lunardays should specify the maximum number of nights from new moon
% which can be utilized to accomplish your scientific goals.  It should
% be a number from 1 (new moon) to 14 (full moon).
%
% \optimaldates should contain the range of OPTIMAL dates.
%
% \acceptabledates should give the range of ACCEPTABLE dates (i.e., you
% would not accept time outside those limits).
%
% To enter the acceptable and optimal date ranges, please use two
% dash-separated dates with 3-letter abbreviations for the month
% (Jan, Feb, Mar, Apr, May, Jun, Jul, Aug, Sep, Oct, Nov, Dec)
% followed by the day.  For example:  \optimaldates{Feb 15 - Apr 23}
%
% If you need to enter two or more ranges of acceptable or optimal dates
% for a single observing run, separate the ranges by commas.  For example:
% \acceptabledates{Oct 7 - Nov 24, Jan 7 - Jan 31}
% It is not necessary to give date ranges based on lunar phase information.
% For instance, if you wish to observe an object in March through May
% within five days of new moon, you may give \lunardays{5} and
% \optimaldates{Mar 1 - May 31} instead of multiple shorter date ranges.
%
% DO NOT remove the \begin{obsrun} and \end{obsrun} blocks.

\begin{obsrun}
\telescope{2.4}
\instrument{OSMOS}
\numnights{7}
\lunardays{14}
\optimaldates{July 1 - July 31}
\acceptabledates{June 1 - July 31}
\end{obsrun}

% \begin{obsrun}
% \telescope{}
% \instrument{}
% \numnights{}
% \lunardays{}
% \optimaldates{}
% \acceptabledates{}
% \end{obsrun}

% \begin{obsrun}
% \telescope{}
% \instrument{}
% \numnights{}
% \lunardays{}
% \optimaldates{}
% \acceptabledates{}
% \end{obsrun}

% You may NOT supply more obsrun blocks.  Three is the limit.

% If there are dates that you cannot use for non-astronomical reasons,
% (i.e., other than moon phase or when your object is up)
% please give the dates by filling in the curly braces in \unusabledates{}.
% Please be brief; \unusabledates is LIMITED TO ONE LINE.

% \unusabledates{}

% In the following "essay question" sections, the delimiting pieces of
% markup (\justification, \feasibility, etc.) act as LaTeX \section*{}
% commands.  If the author wanted to have numbered subsections within
% any of these, LaTeX's \subsection could be used.

% SCIENTIFIC JUSTIFICATION
%
% Give the scientific justification for the proposed observations.
% This section should consist of paragraphs of text (and may include
% EPS figures) that follow the \justification line.
% Try to include an explanation of the overall significance to astronomy.

% In order to include an EPS plot, you should use the LaTeX "figure"
% environment.  The plot file is included with the \plotone{FILENAME}
% command; two side-by-side plot files can be included by typing
% \plottwo{FILENAME1}{FILENAME2}.  Use \caption{} to specify a caption.
% The \epsscale{} command can be used to scale \plotone plots if they
% appear too large on the printed page.
%
% \begin{figure}
% \epsscale{0.85}
% \plotone{sample.eps}
% \caption{Sample figure showing important results.}
% \end{figure}
%
% If you need to rotate or make other transformations to a figure, you may
% use the \plotfiddle command:
% \plotfiddle{PSFILE}{VSIZE}{ROTANG}{HSCALE}{VSCALE}{HTRANS}{VTRANS}
% \plotfiddle{sample.eps}{2.6in}{-90.}{32.}{32.}{-250}{225}
% where HSCALE and VSCALE are percentages and HTRANS and VTRANS are
% in PostScript units, 72 PS units = 1 inch.
%
% If you wish to use the "reference" environment, follow
% the following example:
%
%\begin{references}
%\reference Armandroff \& Massey 1991 AJ 102, 927.
%\reference Berkhuijsen \& Humphreys 1989 A\&A 214, 68.
%\reference Massey 1993 in Massive Stars: Their Lives in the Interstellar
%  Medium (Review), ed. J. P. Cassinelli and E. B. Churchwell, p. 168.
%\reference Massey, Armandroff, \& Pyke 1993, in prep.
%\end{references}

\justification
The gradual decline in angular momentum over time is a well-established
phenomenon in cool (FGK) Main Sequence (MS) stars.
This phenomenon is attributed to magnetic braking: angular momentum is carried
away by magnetised stellar winds and is often used as an astronomical clock.
Stellar rotation periods increase (angular frequency decreases) with age, and
the power-law slope of that relation depends, to first order, only on stellar
mass.
Although several studies have observed, characterised and calibrated the
relation between rotation period and stellar age, the level of dispersion, the
'intrinsic scatter' in these relations is thus unknown.
The precision with which an age can be predicted from a rotation period has
not, therefore been quantified.
Open clusters are useful tools for calibrating stellar properties since (we
assume) their members formed at the same time from the same molecular cloud.
These stellar populations containing tens to thousands of members are
excellent test-beds for age-rotation relations since, being coeval, the stars'
rotation periods should depend only on their mass.
Binaries are minimum-member stellar populations which also, presumably, form
at the same epoch.
These astronomical objects provide additional tests of age-rotation relations
since again, the rotation periods of two stars in a binary system should
depend only on their masses.
In addition, they are often in more isolated environments than open clusters
and are likely to have a different dynamical history.
Semyeong Oh has identified several comoving pairs of stars in the first
release of \gaia data which will be excellent test subjects for the
age-rotation relations.
{\bf We hope to demonstrate that rotation period is an excellent tracer of
stellar age and to quantify the intrinsic scatter in the age-rotation
relations using comoving pairs in the \Gaia\ \TGAS\ catalogue.}


The \gaia spacecraft is an astrometric mission designed to, amongst other
things, measure proper motions and parallaxes for a billion stars in the
galaxy.
The first data release (DR1) contained proper motions (in the plane of the
sky) and parallaxes for two million bright stars with apparent magnitudes of
12 and above.
These data were searched for pairs and groups of stars with similar proper
motions and parallaxes to produce a catalogue of comoving pair and group
candidates.
Some of these candidates will be members of thus unknown, newly discovered
moving groups, others will be members of wide binaries.
Our confidence in the fidelity of these comoving stars' identities is
high --- we have selected only candidates with a less than 1 in 10,000 chance
that they are moving with similar velocities at similar distances by random
conincidence.
There may still be some contamination however, since the \gaia DR1 contained
only proper motions in RA and dec, not in the radial direction.
{\bf We require radial velocities (RVs) in order to improve the confidence in
the membership probabilities of these comoving candidates.}

Of the several thousand comoving pair candidates identified over the entire
sky, around 300 of these are in the \kepler field.
12 stars have rotation periods measured by McQuillan (2013) and we are
currently measuring rotation periods for the remainder.

% FEASIBILITY
%
% Assess the technical and scientific feasibility of the observations.
% This section should consist of text and tables only (no figures)
% following the \feasibility line.
%
% List objects, coordinates, and magnitudes (or surface brightness,
% if appropriate), desired S/N, wavelength coverage and resolution.
% Justify the number of nights requested as well as the specific
% telescope(s), instruments, and lunar phase.  Indicate the optimal
% detector, as well as acceptable alternates.  If you've requested
% long-term status, explain why this is necessary for successful
% completion of the science.

\feasibility

Our targets are bright (11th - 12th mag) and located in the \kepler field
($\alpha = , \gamma = $) which is best observed in July.
Based on the magnitudes of our targets, with average exposure times of 10-12
minutes, we require 7 nights in order to obtain RVs for 150 comoving pair
candidates in the \kepler\ field.
We expect to achieve an RV precision of 15-20 km s$^{-1}$, which will allow us
to assign improved membership probabilities to these comoving pairs and
groups.

% OTHER FACILITIES
%
% Why MDM? If you are using other facilities for this project,
% explain how the MDM observations fit into the scheme of things.
%
% This section should consist of text and tables only (no figures)
% following the \feasibility line.

\whymdm
These are ideal targets for queue observing since they are bright and require
short exposure times.
We are not submitting proposals to other facilities for this project.

% PAST USE
%
% List your allocation of telescope time at MDM during the
% past 3 years, and describe the status of the project (cite
% publications where appropriate).  Mark any allocations of time
% related to the current proposal with a \relatedwork{} command.
% Are your MDM observations achieving their goals?

\thepast
This is the first proposal for MDM lead by PI Ruth Angus.

\end{document}
